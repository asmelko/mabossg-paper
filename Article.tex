%Version 3 October 2023
% See section 11 of the User Manual for version history
%
%%%%%%%%%%%%%%%%%%%%%%%%%%%%%%%%%%%%%%%%%%%%%%%%%%%%%%%%%%%%%%%%%%%%%%
%%                                                                 %%
%% Please do not use \input{...} to include other tex files.       %%
%% Submit your LaTeX manuscript as one .tex document.              %%
%%                                                                 %%
%% All additional figures and files should be attached             %%
%% separately and not embedded in the \TeX\ document itself.       %%
%%                                                                 %%
%%%%%%%%%%%%%%%%%%%%%%%%%%%%%%%%%%%%%%%%%%%%%%%%%%%%%%%%%%%%%%%%%%%%%

%%\documentclass[referee,sn-basic]{sn-jnl}% referee option is meant for double line spacing

%%=======================================================%%
%% to print line numbers in the margin use lineno option %%
%%=======================================================%%

%%\documentclass[lineno,sn-basic]{sn-jnl}% Basic Springer Nature Reference Style/Chemistry Reference Style

%%======================================================%%
%% to compile with pdflatex/xelatex use pdflatex option %%
%%======================================================%%

%%\documentclass[pdflatex,sn-basic]{sn-jnl}% Basic Springer Nature Reference Style/Chemistry Reference Style


%%Note: the following reference styles support Namedate and Numbered referencing. By default the style follows the most common style. To switch between the options you can add or remove �Numbered� in the optional parenthesis. 
%%The option is available for: sn-basic.bst, sn-vancouver.bst, sn-chicago.bst%  
 
%%\documentclass[sn-nature]{sn-jnl}% Style for submissions to Nature Portfolio journals
% \documentclass[sn-basic]{sn-jnl}% Basic Springer Nature Reference Style/Chemistry Reference Style
\documentclass[sn-mathphys-num]{sn-jnl}% Math and Physical Sciences Numbered Reference Style 
%%\documentclass[sn-mathphys-ay]{sn-jnl}% Math and Physical Sciences Author Year Reference Style
%%\documentclass[sn-aps]{sn-jnl}% American Physical Society (APS) Reference Style
%%\documentclass[sn-vancouver,Numbered]{sn-jnl}% Vancouver Reference Style
%%\documentclass[sn-apa]{sn-jnl}% APA Reference Style 
%%\documentclass[sn-chicago]{sn-jnl}% Chicago-based Humanities Reference Style

%%%% Standard Packages
%%<additional latex packages if required can be included here>

\usepackage{graphicx}%
\usepackage{multirow}%
\usepackage{amsmath,amssymb,amsfonts}%
\usepackage{amsthm}%
\usepackage{mathrsfs}%
\usepackage[title]{appendix}%
\usepackage{xcolor}%
\usepackage{textcomp}%
\usepackage{manyfoot}%
\usepackage{booktabs}%
\usepackage{algorithm}%
\usepackage{algorithmicx}%
\usepackage{algpseudocode}%
\usepackage{listings}%

\usepackage{graphicx}
\usepackage{siunitx}

%\usepackage{blindtext}
\usepackage{algorithm}
\usepackage{algpseudocode}
%%%%

%%%%%=============================================================================%%%%
%%%%  Remarks: This template is provided to aid authors with the preparation
%%%%  of original research articles intended for submission to journals published 
%%%%  by Springer Nature. The guidance has been prepared in partnership with 
%%%%  production teams to conform to Springer Nature technical requirements. 
%%%%  Editorial and presentation requirements differ among journal portfolios and 
%%%%  research disciplines. You may find sections in this template are irrelevant 
%%%%  to your work and are empowered to omit any such section if allowed by the 
%%%%  journal you intend to submit to. The submission guidelines and policies 
%%%%  of the journal take precedence. A detailed User Manual is available in the 
%%%%  template package for technical guidance.
%%%%%=============================================================================%%%%

%% as per the requirement new theorem styles can be included as shown below
% \theoremstyle{thmstyleone}%
% \newtheorem{theorem}{Theorem}%  meant for continuous numbers
% %%\newtheorem{theorem}{Theorem}[section]% meant for sectionwise numbers
% %% optional argument [theorem] produces theorem numbering sequence instead of independent numbers for Proposition
% \newtheorem{proposition}[theorem]{Proposition}% 
% %%\newtheorem{proposition}{Proposition}% to get separate numbers for theorem and proposition etc.

% \theoremstyle{thmstyletwo}%
% \newtheorem{example}{Example}%
% \newtheorem{remark}{Remark}%

% \theoremstyle{thmstylethree}%
% \newtheorem{definition}{Definition}%

\raggedbottom
%%\unnumbered% uncomment this for unnumbered level heads

\begin{document}

\title[MaBoSS for HPC environments]{MaBoSS for HPC environments: Implementations of the continuous time Boolean model simulator for large CPU clusters and GPU accelerators}

%%=============================================================%%
%% GivenName	-> \fnm{Joergen W.}
%% Particle	-> \spfx{van der} -> surname prefix
%% FamilyName	-> \sur{Ploeg}
%% Suffix	-> \sfx{IV}
%% \author*[1,2]{\fnm{Joergen W.} \spfx{van der} \sur{Ploeg} 
%%  \sfx{IV}}\email{iauthor@gmail.com}
%%=============================================================%%

% \author*[1,2]{\fnm{First} \sur{Author}}\email{iauthor@gmail.com}

% \author[2,3]{\fnm{Second} \sur{Author}}\email{iiauthor@gmail.com}
% \equalcont{These authors contributed equally to this work.}

% \author[1,2]{\fnm{Third} \sur{Author}}\email{iiiauthor@gmail.com}
% \equalcont{These authors contributed equally to this work.}

% \affil*[1]{\orgdiv{Department}, \orgname{Organization}, \orgaddress{\street{Street}, \city{City}, \postcode{100190}, \state{State}, \country{Country}}}

% \affil[2]{\orgdiv{Department}, \orgname{Organization}, \orgaddress{\street{Street}, \city{City}, \postcode{10587}, \state{State}, \country{Country}}}

% \affil[3]{\orgdiv{Department}, \orgname{Organization}, \orgaddress{\street{Street}, \city{City}, \postcode{610101}, \state{State}, \country{Country}}}

\author[1]{\fnm{Adam} \sur{Šmelko}}\email{smelko@d3s.mff.cuni.cz}

\author[2]{\fnm{Miroslav} \sur{Kratochvíl}}

\author[3,4,5]{\fnm{Emmanuel} \sur{Barillot}}
\author[3,4,5]{\fnm{Vincent} \sur{Noël}}\email{vincent.noel@curie.fr}

\affil*[1]{\orgdiv{Department of Distributed and Dependable Systems}, \orgname{Charles University}, \orgaddress{\city{Prague},  \country{Czech Republic}}}
\affil[2]{\orgdiv{Luxembourg Centre for Systems Biomedicine}, \orgname{University of Luxembourg}, \orgaddress{\city{Esch-sur-Alzette},  \country{Luxembourg}}}
\affil[3]{\orgdiv{Institut Curie}, \orgname{Université PSL, F-75005}, \orgaddress{\city{Paris},  \country{France}}}
\affil[4]{\orgdiv{INSERM}, \orgname{U900, F-75005}, \orgaddress{\city{Paris},  \country{France}}}
\affil[5]{\orgdiv{Mines ParisTech}, \orgname{Université PSL, F-75005}, \orgaddress{\city{Paris},  \country{France}}}


%%==================================%%
%% Sample for unstructured abstract %%
%%==================================%%

\abstract{Computational models in systems biology are becoming more important with the advancement of experimental techniques to query the mechanistic details responsible for leading to phenotypes of interest. In particular, Boolean models are well fit to describe the complexity of signaling networks while being simple enough to scale to a very large number of components. With the advance of Boolean model inference techniques, the field is transforming from an artisanal way of building models of moderate size to a more automatized one, leading to very large models. In this context, adapting the simulation software for such increases in complexity is crucial. 
We present two new developments in the continuous time Boolean simulators: MaBoSS.MPI, a parallel implementation of MaBoSS which can exploit the computational power of very large CPU clusters, and MaBoSS.GPU, which can use GPU accelerators to perform these simulations. }

%%================================%%
%% Sample for structured abstract %%
%%================================%%

% \abstract{\textbf{Purpose:} The abstract serves both as a general introduction to the topic and as a brief, non-technical summary of the main results and their implications. The abstract must not include subheadings (unless expressly permitted in the journal's Instructions to Authors), equations or citations. As a guide the abstract should not exceed 200 words. Most journals do not set a hard limit however authors are advised to check the author instructions for the journal they are submitting to.
% 
% \textbf{Methods:} The abstract serves both as a general introduction to the topic and as a brief, non-technical summary of the main results and their implications. The abstract must not include subheadings (unless expressly permitted in the journal's Instructions to Authors), equations or citations. As a guide the abstract should not exceed 200 words. Most journals do not set a hard limit however authors are advised to check the author instructions for the journal they are submitting to.
% 
% \textbf{Results:} The abstract serves both as a general introduction to the topic and as a brief, non-technical summary of the main results and their implications. The abstract must not include subheadings (unless expressly permitted in the journal's Instructions to Authors), equations or citations. As a guide the abstract should not exceed 200 words. Most journals do not set a hard limit however authors are advised to check the author instructions for the journal they are submitting to.
% 
% \textbf{Conclusion:} The abstract serves both as a general introduction to the topic and as a brief, non-technical summary of the main results and their implications. The abstract must not include subheadings (unless expressly permitted in the journal's Instructions to Authors), equations or citations. As a guide the abstract should not exceed 200 words. Most journals do not set a hard limit however authors are advised to check the author instructions for the journal they are submitting to.}

\keywords{Computational Biology, High Performance Computing, Boolean models}

%%\pacs[JEL Classification]{D8, H51}

%%\pacs[MSC Classification]{35A01, 65L10, 65L12, 65L20, 65L70}

\maketitle

\section{Introduction}

Biological systems are large and complex, and understanding their internal behavior remains critical for designing new therapies for complex diseases such as cancer.
A crucial approach in this endeavor is building computational models from existing knowledge and analyzing them to find intervention points and to predict the efficacy of new treatments.
Many different frameworks have been used to describe biological systems, from quantitative systems of differential equations to more qualitative approaches such as Boolean models.
While the former seems more adapted to represent complex behavior, such as non-linear dependencies, the latter is being increasingly used because of its capability to analyze very large systems.
Many Boolean models have been built to describe biological systems to tackle a variety of problems: from understanding fundamental properties of cell cycle~\cite{faure2006cellcycle,sizek2019boolean} to advanced properties of cancer~\cite{fumia_carcinogenesis_2013,montagud2022prostate}.

Historically, the task of building Boolean models involved reading an extensive amount of literature and summarizing it in a list of essential components and their interactions.
More recently, thanks to advances in databases listing such interactions~\cite{licata2020signor,turei2016omnipath} and to experimental techniques providing information on a bigger number of components, the automatic methods have been designed to infer Boolean formulae from the constraints encoded in the knowledge and the experimental data~\cite{10.1093/bioinformatics/btaa484,chevalier2020synthesis,benevs2023boolean}, allowing construction of large Boolean models.
While this effort faces many challenges, we believe it is a promising way to study the large-scale complexity of biological systems.
However, in order to analyze the dynamic properties of such large Boolean models, we need to develop efficiently scalable simulation tools.

Here, we present adaptations of MaBoSS~\cite{stoll2012continuous, stoll2017maboss} --- a stochastic Boolean simulator that performs estimations of state probability trajectories based on Markov chains -- to modern HPC computing architectures, which provide significant speedups of the computation, thus allowing scrutinization and analysis of much larger boolean models.
The main contributions comprise two new implementations of MaBoSS:
\begin{itemize}
    \item MaBoSS.GPU, a GPU-accelerated implementation of MaBoSS, which is designed to exploit the computational power of massively parallel GPU hardware.
    \item MaBoSS.MPI, a parallel implementation of MaBoSS which can scale to multinode environments, such as large CPU clusters.
\end{itemize}

The source code of the proposed implementations is publicly available at their respective GitHub repositories\footnote{\url{https://github.com/sysbio-curie/MaBoSS.GPU}, \url{https://github.com/sysbio-curie/MaBoSS-env-2.0}}. We also provide the scripts, presented plots, data and instructions to reproduce the benchmarks in the replication package\footnote{\url{https://github.com/sysbio-curie/hpcmaboss-artifact}}.

To showcase the utility of the new implementations, we performed benchmarking on both existing models and large-scale synthetic models. As the main results, MaBoSS.GPU provided over 200\texttimes\ speedup over the current version of MaBoSS on a wide range of models using contemporary GPU accelerators, and MaBoSS.MPI is capable of almost linear performance scaling with added HPC resources, allowing similar speedups by utilizing the current HPC infrastructures.

\section{Background}

\subsection{Boolean signaling models}

% this is very cool but can we simply refer to the maboss paper? or no one took care to formalize this properly yet?

A Boolean signaling model consists of $n$ nodes that are either active or inactive, gaining values 1 or 0 respectively. The \emph{state} of the whole model is represented by a vector $S$ of $n$ Boolean values where $S_i$ represents the value of the $i$-th node. We denote the set of all possible states as $\mathcal{S} = \{0, 1\}^n$; thus $|\mathcal{S}| = 2^n$. 

Interactions in the model are described as transitions between two states. A single state can have multiple transitions to other states with assigned transition probabilities. In turn, a Boolean network is represented as a directed weighted graph $G = (\mathcal{S}, \rho)$, where $\rho: \mathcal{S} \times \mathcal{S} \rightarrow [0, \infty)$ is a transition function generating \emph{transition rates}. For convenience, it holds that:
\[ \rho(S, S') = 0 \iff \text{there is no transition from}\ S\ \text{to}\ S' \]


\subsection{MaBoSS: Markovian Boolean Stochastic Simulator}
MaBoSS simulates the \emph{asynchronous update strategy}, where at most a single node changes its value in each transition --- consequently, a state $S$ can have at most $n$ possible transitions.

To determine the possible transitions, each node follows the \emph{Boolean logic} $\mathcal{B}_i: \mathcal{S} \rightarrow [0, \infty)$, which determines the expected Poisson-process rate of transitioning to the other value. If $\mathcal{B}_i(S) = 0$, then the transition at node $i$ is not allowed in state $S$.
Given this formalization, the simulation can be also viewed as a continuous-time Markov process.

MaBoSS algorithm simulates the above process to produce stochastic \emph{trajectories}: sequences of states $S^0, S^1, \dots, S^k$ and time points $t^0 < t^1 < \dots < t^k$ where $t^0 = 0$ and $S^0$ is the initial state, and for each $i \in \{0, \dots, k-1\}$, $S^i$ transitions to $S^{i+1}$ at time $t^{i+1}$. The simulation ends either by a timeout when reaching the maximal allowed time, or by reaching a fixed point state with no outgoing transitions. The algorithm for a single iteration of the trajectory simulation is given explicitly in Algorithm~\ref{alg:iter}.

\begin{algorithm}
\caption{A single iteration of the MaBoSS simulation of a trajectory, given the state $S$ and time $t$.}
\label{alg:iter}
\begin{algorithmic}[1]
\Procedure{TrajectorySimulationStep}{S, t}
\State $\rho_1 \gets \mathcal{B}_1(S), \dots, \rho_n \gets \mathcal{B}_n(S)$ \Comment compute transition rates
\State $r \gets \text{random}([0, \sum_{i=1}^n \rho_i))$
\State $i \gets \min_i \sum_{j=1}^{i-1} \rho_j \leq r < \sum_{j=1}^{i} \rho_j$ \Comment select a node for flipping
\State $S'\gets S\ \text{with the $i$-th bit flipped}$
\State $u \gets \text{random}([0, 1])$
\State $\delta t \gets -\frac{\ln u}{\sum_{i=1}^n \rho_i}$
\State \textbf{return} $(S', t + \delta t)$
\EndProcedure
\end{algorithmic}
\end{algorithm}

To obtain a good overview of the Markovian process, we apply the Gillespie stochastic simulation algorithm\cite{gillespie1976general} (or kinetic Monte-Carlo\cite{young1966monte}
) on the Boolean state space. Multiple trajectories are generated and aggregated in compound trajectory statistics. Commonly obtained statistics include:
\begin{itemize}
    \item \emph{Network state probabilities on a time window} --- Trajectory states are divided by their transition times into time windows based on the time intervals specified by a window size. For each window, the probability of each state is computed as the duration spent in the state divided by the window size. The probabilities of the corresponding windows are then averaged across all subtrajectories.
    \item \emph{Final states} --- The last sampled states from the trajectories are used to compute a final state distribution.
    \item \emph{Fixed states} --- All reached fixed points are used to compute a fixed state distribution.
\end{itemize}

To maintain the brevity in the statistics, MaBoSS additionally allows marking some nodes \emph{internal}.
This is useful because nodes that are not ``interesting'' from the point of final result view occur quite frequently in Boolean models, and removing them from statistics computation often saves a significant amount of resources.

\subsection{Computational complexity of parallel MaBoSS algorithm}

\subsubsection{Simulation complexity}

We estimate the time required to simulate $c$ trajectories as follows:
For simplification, we assume that a typical Boolean logic formula in a model of $n$ nodes can be evaluated in $\mathcal{O}(n)$ (this is a very optimistic but empirically valid estimate). With that, the computation of all possible transition rates (Algorithm~\ref{alg:iter}, line $2$) can be finished in $\mathcal{O}(n^2)$. The selection of the flipping bit (Algorithm~\ref{alg:iter}, line $4$) can be finished in $\mathcal{O}(n)$, and all other parts of the iteration can finish in $\mathcal{O}(1)$. In total, the time complexity of one iteration is $\mathcal{O}(n^2)$. If we simulate $c$ trajectories with an upper bound of trajectory length $u$, the simulation time is in $\mathcal{O}(c \cdot u \cdot n^2)$.

In an idealized PRAM (parallel random access machine~\cite{fortune1978parallelism}) model with infinite parallelism, we can optimize the algorithm in the following ways:
\begin{itemize}
    \item Given $c$ processors, all trajectory simulations can be performed in parallel, reducing the time complexity to $\mathcal{O}(u \cdot n^2)$. (Note that this does not include the results aggregation. See \emph{Statistics aggregation} section for further description.)
    \item With $n$ processors, the computation of transition rates in the simulation can be done $\mathcal{O}(n)$ time, and the selection of the flipping bit can be done in $\mathcal{O}(\log{n})$ time using a parallel prefix sum, giving $\mathcal{O}(n)$ time for a single iteration.
\end{itemize}
Thus, using a perfect parallel machine with $c \cdot n$ processors, the computation time can be reduced to $\mathcal{O}(u \cdot n)$. Notably, the $\mathcal{O}(u)$ simulation steps that must be performed serially remain a major factor in the whole computation time.

\subsubsection{Statistics aggregation}

The aggregation of the statistics from the simulations is typically done by updating a shared associative structure indexed by model states, differing only in update frequency between the three kinds of collected statistics.

If the associative structure is implemented as a hashmap, the updates can be done in $\mathcal{O}(1)$ for a single process. With multiple processors, the algorithm may hold partial versions of the hashmap for each processor, and aggregate all of them at the end of the computation, which can be done in $\mathcal{O}(\log{c} \cdot m)$ using $c$ processors, assuming the maximal size of statistic to be $m$.

As an interesting detail, the hash structures pose a surprising constant-factor overhead. In networks where most nodes are internal, the hash map may be replaced by a fixed-size multidimensional array that holds an element for all possible combinations of external node values (basically forming a multidimensional histogram). We discuss the impact of this optimization in the next section.


\subsubsection{MaBoSS CPU Implementation}
MaBoSS is multi-core...

\section{Implementation}\label{sec:implementation}

\subsection{MaBoSS.GPU}

\subsubsection*{Simulation}

In the CPU version of MaBoSS, the simulation part is the most computationally demanding part, with up to 80\% of MaBoSS runtime spent by just evaluating the Boolean formulae (the exact number depends on the model). The original formula evaluation algorithm in MaBoSS used a recursive traversal of the expression tree, which (apart from other issues) causes memory usage patterns unsuitable for GPUs: the memory required per each core is not achievable in current GPUs, and there are typically too many cache misses~\cite{karlsson2000prefetching}.

There are multiple ways to optimize the expression trees for GPUs: One may use a linked data structure that is more cache-friendly such as the van Emde Boas tree layout~\cite{van1975preserving}, or perhaps represent the Boolean formulae as a compact continuous array, or convert it to CNF or DNF (conjunctive or disjunctive normal form) bitmasks that can be easily evaluated by vector instructions. We decided to leave the exact representation choice on the compiler, by encoding the expressions as direct code and using the runtime compilation of GPU code~\cite{nvrtc}. In such an approach, the application reads the model files, writes the formulae as functions in CUDA C++ language, compiles them using the NVIDIA runtime compiler, and finally runs the simulation on GPU --- all without user intervention.

Using this technique, the Boolean formulae are compiled as functions into a native binary code, which is directly executed by the GPU. As the main advantage, the formulae are encoded in the instructions, preventing unnecessary fetches of the encoded formulae from other memory. At the same time, the compiler may apply a vast spectrum of optimizations on the Boolean formulae, including case analysis and shortcutting, again resulting in faster evaluation.

A possible drawback of the runtime compilation stems from the relative slowness of the compiler --- for small models, the total execution time of MaBoSS.GPU may be easily dominated by the compilation.

The work distribution was chosen to be one trajectory simulation per GPU thread. Due to the involved implementation complexity, we avoided optimization of the computation of individual trajectories by splitting the Boolean function evaluation into multiple threads (thus missing the factor of $n$ threads from the asymptotic analysis). While such optimization might alleviate some cache pressure and thus provide significant performance improvements, we leave its exploration to future work.

\subsubsection*{Statistics aggregation}

For optimizing the statistics aggregation, MaBoSS.GPU heavily relies on the fact that the typical number of non-internal nodes in a real-world MaBoSS model rarely exceeds 10 nodes, regardless of the size of the model. This relatively low number of states generated by non-internal nodes allows us to materialize the whole statistics structure (called ``histogram'') as a fixed-size array (rarely exceeding $2^{10}$ elements).

This approach allows us to avoid storing the states as the keys and gives a simple approach that can map the state to the histogram index using simple bit masking and shifting instructions. Further, we use several well-known GPU histogram update optimizations to improve the performance, including shared memory privatization and atomic operations.

\subsection{MaBoSS.MPI}

MaBoSS.MPI is a straightforward extension of the original MaBoSS CPU code to the MPI programming interface.
Briefly, each MPI node is assigned to simulate the same number of trajectories (up to a remainder). These are further uniformly distributed among the CPU cores of the node, each thread progressively collecting the results into a privatized hashmap-based statistics aggregation structure. 

Once all trajectory simulations are finished and the statistics are computed for each thread, the intermediate data are reduced into the final result using MPI collective operations.


% MaBoSS' CPU implementation already offered parallelism, allowing to use multiple CPU cores to compute the multiple trajectories. However, this implementation stays limited to a single machine CPU, and its scalability suffers from a very intensive memory usage. Furthermore, with the increasing number of nodes in Boolean models, the need for very large number of trajectories leads to large simulation time. 
% To tackle this, we decided to extend this implementation and add another layer of parallelism using MPI. In this implementation, the simulation of the trajectories is not only distributed on multiple cores, but on the multiple cores of multiple MPI nodes. By using this implementation, we are able to exploit the large CPU clusters available in HPC centers. 
% The implementation of this MPI implementation is straightforward, and consists in distributing the invidivual trajectories even further, first on MPI nodes and then on the CPU cores of the MPI node. Once the trajectories are simulated, the statistics are first computed for each MPI nodes, and then the global results is computer from the results of each MPI nodes. 


\section{Results}

To evaluate the impact of the implemented optimizations, we present the results of performance benchmarks for MaBoSS.GPU and MaBoSS.MPI by comparing their runtimes against the original CPU implementation. To obtain a comprehensive overview of achievable results, we used both real-world models and synthetic models with varying sizes.

\subsection{Benchmarking Methodology}

For the benchmarks, we used $3$ real-world models of $10$, $87$ and $133$ nodes (\textsc{cellcycle}~\cite{faure2006cellcycle}, \textsc{sizek}~\cite{sizek2019boolean} and \textsc{Montagud}~\cite{montagud2022prostate}). In order to test the scalability of the GPU and MPI implementation, we also created several synthetic models with up to $1000$ nodes. Synthetic models were designed in a way such that the length of each trajectory is predictable, and the models have no stable states. Also, the number of non-internal nodes was kept low (10 nodes) to enable the usage of the histogram optimization. The synthetic models together with their Python generator are available in the replication package.

The GPU implementation benchmarks were run on a datacenter-grade NVIDIA Tesla A100 GPU and a consumer-grade NVIDIA RTX 3070 Laptop GPU. The CPU implementation benchmarks were run on a 32-core Intel Xeon Gold 6130 CPU with multithreading. The CPU implementation was compiled with GCC 13.2.0, and the GPU implementation was compiled with CUDA 12.2. Each measurement was repeated $10$ times, and the average runtime was used as the final result.

The MPI implementation benchmarks were run on the MareNostrum 4 supercomputer\footnote{\url{https://www.bsc.es/marenostrum/marenostrum}}.

\subsection{Performance of MaBoSS.GPU}

\begin{figure}
    \centering
    \includegraphics[width=\linewidth]{plots/real.pdf}
    \caption{Wall time comparison of MaBoSS and MaBoSS.GPU on real-world models. Each model is simulated with 1 million trajectories.}
    \label{fig:real}
\end{figure}

In Figure~\ref{fig:real}, we compare the wall time of the CPU and GPU implementations on real-world datasets. The GPU implementation is faster than the CPU implementation on all models, and the speedup shows to be more significant on the models with more nodes and longer trajectories. On the \textsc{Montagud} model with 133 nodes, but a relatively short average trajectory, we achieve $145\times$ speedup. On a slightly smaller \textsc{sizek} model with a longer average trajectory, the speedup is up to $326\times$. 

It is worth noting that the datacenter GPU performs worse than the laptop GPU. Both devices are bottlenecked by the runtime compilation of the Boolean formulae, however, NVIDIA A100 spends on average around $300$ms more on the compilation step. Subtracting the compilation time, A100 is faster for all models. We did not spend time finding the root cause of this discrepancy since the value is negligible and the following benchmarks show that the runtime compilation overhead quickly disappears with increasing model size.

\begin{figure}
    \centering
    \includegraphics[width=\linewidth]{plots/nodes.pdf}
    \caption{Wall time comparison of MaBoSS and MaBoSS.GPU on synthetic models with sizes ranging from $10$ to $1000$ nodes (x-axis). Each model is simulated with 1 million trajectories. The two panels differ by the inclusion of the runtime compilation of the model logic, showing its impact on total run time.}
    \label{fig:synth}
\end{figure}

Figure~\ref{fig:synth} shows much finer performance progression on synthetic models. We observed that the CPU variant starts to progress steeper at around the $100$ nodes boundary. We assume that the implementation hits the cache size limit, and the overhead of fetching the required data from the memory becomes dominant. The same can be observed in the GPU variant later at around $200$ nodes. Expectably, the cache-spilling performance penalty is much more significant on GPUs. Overall, the results suggest that the optimization of dividing transition rate computations among multiple threads, as mentioned in \emph{Implementation} section, may provide a better speedup for bigger models, as it alleviates the register and cache pressure.

\begin{figure}
    \centering
    \includegraphics[width=\linewidth]{plots/nodes-compilation-big-NVIDIA Tesla A100 (MaBoSS.GPU).pdf}
    \caption{The ratio of time spent in the runtime compilation of the Boolean formulae in relation to the total runtime, simulating models with varying numbers of nodes, trajectories, and formula lengths.}
    \label{fig:comp}
\end{figure}

Additionally, Figure~\ref{fig:synth} shows the total runtime of the GPU implementation including the runtime compilation step. Comparing the panels, we observe that the relative runtime compilation overhead quickly disappears with increasing model size. Figure~\ref{fig:comp} shows the results of more detailed benchmarks for this scenario, as run on the NVIDIA Tesla A100 GPU. We observed that the compilation time is linearly dependent on the number of nodes and formula lengths (measured in the number of occurring nodes), which can be simply explained by the fact that these model properties extend source files that need to be compiled by a linear factor. Notably, as soon as the simulation becomes more computationally complex (e.g., by increasing the number of nodes or simulated trajectories), the compilation time becomes relatively negligible even for models with unrealistically long formulae. This suggests that the runtime compilation is a viable optimization methodology also for much larger models.

\subsection{Performance of MaBoSS.MPI}


\begin{figure}%[tbhp]
\centering
\includegraphics[width=3.25in]{plots/sizek_mpi.pdf}
\caption{Scalability results of MPI implementation on Sizek model with up to 192 MPI nodes and 20 cores per node, summing up to 3840 cores.}
\label{fig:sizek_results}
\end{figure}

Figure~\ref{fig:sizek_results} shows the efficiency of the MaBoSS.MPI implementation on the \textsc{sizek} model. We ran multiple suites, ranging from a single MPI node up to 192 nodes, each running 20 cores. We can observe a close-to-linear speedup of up to 64 MPI nodes (1280 cores), and a plateau for larger suites (Figure~\ref{fig:sizek_results}, green). This can be explained by hitting an expectable bottleneck in parallelization overhead and MPI communication cost when the problem is divided into too many small parts.

\begin{figure}%[tbhp]
\centering
\includegraphics[width=3.25in]{plots/synth_mpi_speedup.pdf}
\caption{Speedup scaling of MPI implementation on a synthetic model with 1000 nodes, running on up to 192 MPI nodes with 32 cores per MPI node, summing up to 6144 cores.}
\label{fig:synthetic_results}
\end{figure}

To stress the scalability of the implementation, we also used a synthetic model with 1000 nodes. We simulated this model on 32 cores per MPI node, on 1 to 192 nodes (32 to 6144 cores). The obtained speedups are summarized in Figure~\ref{fig:synthetic_results}. Using this configuration, the simulation time decreases from 20 hours on 1 MPI node to 430 seconds on 192 nodes. As expected, the plateau in the speedup was not observed in simulations that involve larger models.

% We simulated a cell cycle model by Sizek et al.\cite{sizek2019boolean} (Figure~\ref{fig:sizek_results}). A comparison of execution times of MaBoSS v2.5.3 and v2.4.0 using 40 cores and one million individual simulations showed a considerable (3×) speed-up between the pre- and post-optimisation versions, which is mostly due to optimised memory usage (figure 2, purple vs red). We used the same test with the MPI implementation of MaBoSS, using 20 cores for each MPI node, from 1 MPI node up to 192 MPI nodes. We can observe a very good speedup (>50x) up to 64 MPI nodes (1.28k cores), and a plateau for larger setups (Figure 7, green). This is probably due the very short simulation time (10s) for that setup and the minimum overhead for such a parallel simulation, and needs testing on larger simulations.


% In order to explore larger model sizes, we created a synthetic model with 1000 nodes (Figure~\ref{fig:synthetic_results}). We simulated this model on 32 cores per MPI node, on 1 to 192 MPI nodes (32 to 6144 cores). With this dataset, the simulation time goes from 20 hours on 1 MPI node to 430 seconds on 192 nodes. As we hypothesised, with larger simulation times, a decline in the speedup was not observed as in the previous example 
% The MaBoSS benchmarking measurements described in this section were performed on MareNostrum 4.

% \begin{figure}%[tbhp]
% \centering
% \includegraphics[width=.8\linewidth]{plots/large_model.pdf}
% \caption{Scalability results for synthetic model with 1000 nodes}
% \label{fig:synthetic_results}
% \end{figure}

%\begin{figure}%[tbhp]
%\centering
%\includegraphics[width=.8\linewidth]{plots/Figure_1}
%\caption{Placeholder image of Iris with a long example caption to show justification setting.}
%\label{fig:computerNo}
%\end{figure}

\section{Conclusions}

In this work, we presented two new implementations of MaBoSS tool, a continuous time Boolean model simulator, both of which are designed to enable utilization of the HPC computing resources: MaBoSS.GPU is designed to exploit the computational power of massively parallel GPU hardware, and MaBoSS.MPI enables MaBoSS to scale to many nodes of HPC clusters via the MPI framework. We evaluated the performance of these implementations on real-world and synthetic models and demonstrated that both variants are capable of providing significant speedups over the original CPU code. The GPU implementation shows 145--326\texttimes\ speedup on real-world models, and the MPI implementation delivers a close-to-linear strong scaling on big models.

Overall, we believe that the new MaBoSS implementations enable simulation and exploration of the behavior of very large, automatically generated models, thus becoming a valuable analysis tool for the systems biology community.

\subsection{Future work} 

During the development, we identified several optimization directions that could be taken by researchers to further scale up the MaBoSS simulation approach.

Mainly, the parallelization scheme used in MaBoSS.GPU could be enhanced to also parallelize over the evaluation of Boolean formulae. To avoid GPU thread divergence, this would however require a specialized Boolean formula representation, entirely different from the current version of MaBoSS; likely even denying the relative efficiency of the use of runtime compilation. On the other hand, this optimization might decrease the register pressure created by holding the state data, and thus increase the performance on models with thousands of nodes.

In the long term, easier optimization paths might lead to sufficiently good results: For example, backporting the GPU implementation improvements back to the MaBoSS CPU implementation could improve the performance even on systems where GPU accelerators are not available. Similarly, both MaBoSS.GPU and MaBoSS.MPI could be combined into a single software that executes distributed GPU-based analysis over multiple MPI nodes, giving a single high-performance solution for extremely large problems.

\section*{Availability and requirements}
% flex and bison are linux only :/
\begin{itemize}
	\item Project name: MaBoSS.GPU
	\item Project home page: \url{https://github.com/sysbio-curie/MaBoSS.GPU}
	\item Operating system(s): Platform independent
	\item Programming language: C++, CUDA
	\item Other requirements: Flex, Bison, CMake $>=$ 3.18, Cuda toolkit $>=$ 12.0
	\item License: MIT
\end{itemize}

\begin{itemize}
	\item Project name: MaBoSS.MPI
	\item Project home page: \url{https://github.com/sysbio-curie/MaBoSS}
	\item Operating system(s): Platform independent
	\item Programming language: C++
	\item Other requirements: Flex, Bison
	\item License: BSD3-clause
\end{itemize}


\backmatter

\section*{Declarations}

% Some journals require declarations to be submitted in a standardised format. Please check the Instructions for Authors of the journal to which you are submitting to see if you need to complete this section. If yes, your manuscript must contain the following sections under the heading `Declarations':

\subsection*{Availability of data and materials}

The scripts, presented plots, data and instructions to reproduce the benchmarks are available on GitHub~\cite{adam_smelko_2024_10853153}.

\subsection*{Competing Interests}
The authors declare that they have no competing interests.

\subsection*{Funding}
The research leading to these results has received funding from the European Union's Horizon 2020 Programme under the PerMedCoE Project (\url{http://www.permedcoe.eu}), grant agreement n\textsuperscript{o}~951773.
% This work was supported by the European Commission under the PerMedCoE project [H2020-ICT-951773]
The project was partially supported by Charles University, SVV project number~260698.

\subsection*{Acknowledgements}
We thank Laurence Calzone and Gautier Stoll for their guidance and fruitful discussions.


%%===================================================%%
%% For presentation purpose, we have included        %%
%% \bigskip command. Please ignore this.             %%
%%===================================================%%
% \bigskip
% \begin{flushleft}%
% Editorial Policies for:

% \bigskip\noindent
% Springer journals and proceedings: \url{https://www.springer.com/gp/editorial-policies}

% \bigskip\noindent
% Nature Portfolio journals: \url{https://www.nature.com/nature-research/editorial-policies}

% \bigskip\noindent
% \textit{Scientific Reports}: \url{https://www.nature.com/srep/journal-policies/editorial-policies}

% \bigskip\noindent
% BMC journals: \url{https://www.biomedcentral.com/getpublished/editorial-policies}
% \end{flushleft}

% \begin{appendices}

% \section{Section title of first appendix}\label{secA1}

% An appendix contains supplementary information that is not an essential part of the text itself but which may be helpful in providing a more comprehensive understanding of the research problem or it is information that is too cumbersome to be included in the body of the paper.

%%=============================================%%
%% For submissions to Nature Portfolio Journals %%
%% please use the heading ``Extended Data''.   %%
%%=============================================%%

%%=============================================================%%
%% Sample for another appendix section			       %%
%%=============================================================%%

%% \section{Example of another appendix section}\label{secA2}%
%% Appendices may be used for helpful, supporting or essential material that would otherwise 
%% clutter, break up or be distracting to the text. Appendices can consist of sections, figures, 
%% tables and equations etc.

% \end{appendices}

%%===========================================================================================%%
%% If you are submitting to one of the Nature Portfolio journals, using the eJP submission   %%
%% system, please include the references within the manuscript file itself. You may do this  %%
%% by copying the reference list from your .bbl file, paste it into the main manuscript .tex %%
%% file, and delete the associated \verb+\bibliography+ commands.                            %%
%%===========================================================================================%%

\bibliography{bibliography}% common bib file
%% if required, the content of .bbl file can be included here once bbl is generated
%%\input sn-article.bbl


\end{document}
